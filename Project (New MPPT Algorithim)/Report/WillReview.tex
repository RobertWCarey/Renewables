\documentclass[]{article}
\usepackage{titlesec}
\usepackage{graphicx}
\usepackage{caption}
\usepackage{amsmath}
\usepackage{color}
\usepackage{cleveref}
\usepackage{float}
\usepackage{siunitx}
\usepackage{fancyhdr}
\usepackage[style=authoryear-ibid,backend=biber]{biblatex}
\usepackage[left = 1.5cm, right = 1.5cm, top = 2cm, bottom = 2cm]{geometry}

\addbibresource{WillReferences.bib}
\pagestyle{fancy}

%\hypersetup{pdfborder={0 0 0}}
\sisetup{scientific-notation=engineering}


\begin{document}	
	\section{Fuzzy Logic MPPT}
		\subsection{What particular issue of MPPT has this paper \parencite{harrabi2017comparative} addressed?}
			Fuzzy logic MPPT addresses the power fluctuation issue which is a large downside to MPPT methods such as Perturb and Observe (P\&O) and Incremental Conductance (IC) which stems from their oscillation about the maximum power point (MPP) of the system. It also addresses  the inability to compensate for sudden insolation change which is an issue with methods such as Short Circuit Pulse or Open-Circuit Voltage. Due to the large number of non-linearities and changing dynamics that exist as part of a solar PV system, more complex control systems, such as using Fuzzy Logic Control, have been developed to overcome these issues.
		\subsection{What is the algorithm of the proposed MPPT?}
			The algorithm changes the inputs of the system, change in PV voltage and change in PV output power, and the output of the system, the DC-DC converter's duty cycle, into fuzzy variables. Membership functions are also developed using a number of methods, including trial and error. Based on the fuzzy values and the membership functions of each value, a fuzzy rule is applied and the output is inferred. The value of the output is generated by aggregating any rules which have been used, and their inferred outputs, using either a centre of area method or by using a weighted average.
		\subsection{How does this MPPT algorithm address the issue?}
			By making the inputs and outputs of the system fuzzy variables, the control system is better equipped to handle sudden changes in insolation and the inherent non-linearities in a solar PV system. It also does not require an in-depth knowledge of the system model, meaning it is more robust.
		\subsection{What are the improvement shown from this MPPT?}
			Fuzzy logic control algorithms require no previous knowledge of the system and are more robust than more conventional MPPT algorithms, while also decreasing oscillation about the operating point of the system and responding faster to changing inputs and atmospheric conditions. Due to these improvements, the efficiency of the system is also increased.
		\subsection{Are there any potential issues with this proposed MPPT?}
			Although it is considered easier to implement than some other control systems, such as using particle swarm optimisation or a genetic algorithm, its comparative complexity compared to more conventional MPPT methods such as P\&O are a barrier to its widespread implementation, as it requires more computational power and performance, as well as being more complex to implement.
	\newpage
	\printbibliography
\end{document}          
